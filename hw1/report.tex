\documentclass{article}

\usepackage{fancyhdr}
\usepackage{mathtools}
\usepackage{amsmath}

\pagestyle{fancy}

\lhead{Robin Touche 900610-3270}
\rhead{robint@student.chalmers.se}

\begin{document}
\chapter{Homework 1}
\setcounter{section}{1}
\subsection{}

To find $\sigma$ we need to try to maximise the likelihood function
$\mathcal{L}$ given as the probability density function $f(x_1, x_2, ... , x_n
\vert \theta)$ where $\theta = (\mu, \sigma^2)$.

The probability density function is given by:

\begin{align}
  f(\mathbf{x} \vert \mu , \sigma^2) = \frac{1}{2\pi^{p / 2} \sigma}
  \cdot e^{\left( - \frac{1}{2} \left( \mathbf{x} - \mu \right) ^T
  \sigma^2 \left(\mathbf{x} - \mu \right) \right)}
\end{align}

The likelihood function can be simplified to:

\begin{align}
  \mathcal{L}(\theta ; \mathbf{x}) &= \prod_{i = 1}^{n}f(x_i \vert \theta) =\\
  \notag &= \left( \frac{1}{\sigma\sqrt{2\pi}^{p}} \right) ^n
    \cdot e^{\left( - \frac{1}{2} \sum_{i = 1}^{n}
    \notag \left( x_i - \mu \right) ^T \sigma^2 \left(x_i - \mu \right) \right)}
\end{align}

To get rid of the annoying exponent we can instead work with the natural
logarithm of $\mathcal{L}$.

\begin{align}
  ln(\mathcal{L}) = - \frac{n}{2} ln(2\pi) - ln(\sigma)
    - \frac{1}{2} \sum_{i = 1}^{n}
    \left( x_i - \mu \right) ^T \sigma^2 \left(x_i - \mu \right)
\end{align}

Since we want to maximise this we can find $\sigma$ s.t. the derivative
$\frac{d}{d\sigma}ln(\mathcal{L}) = 0$.

\begin{align}
  \frac{d}{d\sigma}ln(\mathcal{L}) = -\frac{n}{\sigma} + \frac{1}{\sigma^3}
  \sum_{i = 1}^{n} \left( x_i - \mu \right) ^T \left(x_i - \mu \right) \Rightarrow \\
  \sigma^2 = \frac{1}{n} \sum_{i = 0}^{n - 1}
    \left( x_i - \hat{x} \right) \left(x_i - \hat{x} \right)^T
\end{align}

where $\hat{x}$ is the expected value of $x$.

\subsection{}

\setcounter{section}{2}
\setcounter{subsection}{0}
\subsection{}
\paragraph{a)}

By using the logarithm likelihiid function as above we can estimate $\mu$ as:

\begin{align}
  \mu = \frac{1}{n} \sum_{i = 1}^{n} x_n
\end{align}

We can use this $\nu$ to calcualte $\sigma$ as:

\begin{align}
  \sigma = \frac{1}{n} \sum_{i = 1}^{n} (x_n - \mu)(x_n - \mu)^T
\end{align}

\paragraph{b, c)}
See code

\paragraph{d)}

\end{document}
