\documentclass{article}

\usepackage{fancyhdr}
\usepackage{mathtools}
\usepackage{amsmath}
\usepackage{amsfonts}

\pagestyle{fancy}

\author{
  Robin Touche \\
  \and
  Fredrik Bredmar
}

\title{Machine Learning Homework 3}

\begin{document}

\maketitle

\setcounter{section}{1}
\subsection{}
\paragraph{a)}

Since we have a normal distribution the random variable can take on any real
value. Thus the state space is $\mathbb{R}^2$.

\paragraph{b)}

% $(X_1, X_2 = x_2) = \mathcal{N} \left( \mu_1 + \frac{\sigma_1}{\sigma_2}\rho \left( x_2 - \mu_2 \right), \left( 1 - \rho^2 \right) \sigma_1^2 \right)$ 

\paragraph{c)}

TODO

For each iteration we calculate the conditional probability of $(X_1 \vert X_2
= x_2)$, which becomes a normal distribution. Then we take a random sample from
this distribution which we will call $x_1$.

The next iteration we instead calculate and sample from $(X_2 \vert X_1 = x_1)$
with the $x_1$ we just aquired and set this sample as $x_2$ for the next
iteration where we repeat the prevoius step.

\setcounter{section}{2}
\subsection*{2.1}

See code.

\subsection*{2.2}

\paragraph{a}

Equation one in the assignment description is the conditional posterior distribution that looks as follows.
$$ P(Z_i = k | Z_{\neg i}, W) \propto P(w_i | z_i = k, z_{\neg i}, w_{\neg i})P(z_i = k | z_{\neg i })$$

So the first term on the right hand side is the likelihood function and the second term is the prior. Then by integrating the first term over $\beta$ we end up with;

$$P(w_i | z_i = j, z_{\neg i}, w_{\neg i}) = \frac{ N^{w_i}_{k,\neg i} + \eta }{V\eta + \sum_{w^{\theta}=1}^V N^{w^\theta}_{k, \neg i}}  $$

which is the first part of equation one in the assignment description. In similar way but when integrating over $\theta$ we get the following expression for term two in equation one.

$$P(z_i = k | z_{\neg i}) = \frac{\alpha + M^{k}_{d,\neg i}}{T\alpha + \sum_{k^\theta=1}^T M^k_{d, \neg i} } $$

As we see we end up with equation two if we replace the right hand side with the corresponding equations stated above.
\paragraph{b}

$\beta$ represents what words related to each topic. For a given topic k,
$\beta_k$ is a distribution of how likely each word is to appear with the
topic. Likewise, for a given word, $\beta$ represents what topics this word was
assigned to and how often.

In the same vein $\theta$ represents the distribution of topics in each
document and vice versa.

\paragraph{c}
\paragraph{d}
\paragraph{e}
\paragraph{f}

\end{document}
