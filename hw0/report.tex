\documentclass{article}

\usepackage{fancyhdr}
\usepackage{mathtools}

\pagestyle{fancy}

\lhead{Robin Touche 900610-3270}
\rhead{robint@student.chalmers.se}

\begin{document}
\setcounter{section}{1}
\chapter{Homework 0}
\subsection{}

Bayes' rule states that $P(A\mid B) = \frac{P(A)P(B\mid A)}{P(B)}$. Let A be the
probability of me having cancer and let B be the probability of me testing
positive.

So to calcuate the probability of me having cancer given that I test positive
($P(A\mid B)$) we must calculate $P(A)$, $P(B\mid A)$ and $P(B)$.  We know
$P(A)$ (me having cancer) and $P(B\mid A)$ (me testing positive given that I
do, in fact, have cancer) from the assignment. 0.0001 and 0.99 respectively.

To calculate $P(B)$ we take the probability of testing positive times the
probability of having cancer plus the probability of me testing a false
positive. We can add these probabilities since we know that the events are
totally disjunct (I can't both have cancer and not have it).

\begin{equation*}
0.99 \cdot 0.0001 + 0.01 \cdot 0.9999 = 0.0100989
\end{equation*}

So the final probability becomes:

\begin{equation*}
  P(A\mid B) = \frac{P(A)P(B\mid A)}{P(B)} =
  \frac{0.0001 \cdot 0.99}{0.0100989} \approx 0.0098 \approx 1\%
\end{equation*}

\subsection{}



\end{document}
